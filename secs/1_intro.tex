\section{Introducción}
\subsection{Máquinas térmicas}
Una maquina térmica es un conjunto de elementos mecánicos que permiten intercambiar energía, la cual usando el primer principio de la termodinámica, toma calor de una fuente y logra convertir parte de ella en
trabajo,esto en un proceso cíclico.

La eficiencia de este sistema es definida a partir del trabajo(W) que realiza y el calor que recibe de la fuente($Q_H$), como:

\begin{equation}
    \eta = \frac{W}{Q_H}=\frac{P_W}{P_H}
\end{equation}
Donde $P_H=P_W+P_C$

\subsection{Principio cero de la termodinámica}
Este principio establece que dos o más cuerpos en contacto que se encuentran a distinta temperatura alcanzan, pasado un tiempo, a través de un intercambio de energía, el equilibrio térmico; este intercambio depende del tipo de trasformación que ha experimentado el sistema y a esta transferencia es a la que se le llama calor, entonces los cuerpos no almacenan calor sino energía interna. La expresión que relaciona la cantidad de calor $Q[J]$ que intercambia una cierta sustancia de masa $m[g]$ y calor específico $c_e[J/g°C]$ con la variación de temperatura $\Delta T[°C]$ que experimenta es:
\begin{equation}
    Q=mc_e\Delta T
\end{equation}

\subsection{Segundo principio de la termodinámica}
Este principio establece la irreversibilidad de los fenómenos físicos, especialmente durante el intercambio de calor. Es un principio de la evolución que fue enunciado por primera vez por Sadi Carnot en 1824. Después ha sido objeto de numerosas generalizaciones y formulaciones sucesivas por Clapeyron (1834), Clausiuos(1850), Lord Kelvin, Ludwig Boltzmann en 1873 y Max Planck.

Según Planck, es imposible construir un motor que, trabajando según un ciclo completo, no produzca otro efecto que elevar un peso y enfriar una fuente caliente.

\subsection{Ciclo de Carnot}
Este es un ciclo termodinámico que se produce una maquina cuando trabajo absorbiendo una cantidad de calor de una fuente de mayor temperatura cede calor a una de menor temperatura produciendo un trabajo. 

Este ciclo está conformado por 4 procesos, el primero es una expansión isoterma ( A $$ B), después una expansión adiabática(B ! C), después una comprensión isoterma (C ! D) y por ultimo una compresión adiabática ( D ! A). La máxima eficiencia de este ciclo está dada por: