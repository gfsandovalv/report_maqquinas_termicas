\section{Descripción experimental}

El montaje usado consiste en un módulo Peltier montado en una base de aluminio, la cual se sumergió en una nevera de icopor llena de agua; el propósito del agua es servir como reservorio térmico. 

El primer montaje que se usó fue el del módulo como generador termoeléctrico. En la parte superior del módulo hay una resistencia de valor $R_c$ adherida, a la cual se le aplica una diferencia de potencial de modo que se establece una corriente. Esta resistencia se calentará, consecuentemente calentando el módulo, este calor aplicado al módulo da lugar a una diferencia de potencial que se mide haciendo uso de una resistencia de carga $R_L$ conectada a los bornes. 

El segundo montaje fue el del módulo como refrigerador o bomba térmica. En este caso ya no se usa la resistencia de calentamiento, y en cambio, la diferencia de potencial esta vez se aplica al módulo directamente. Al realizar esta conexión, la temperatura en la parte superior del módulo comenzará a disminuir, mientras que el calor absorbido del aire se deposita en el agua, que por ser una masa considerablemente grande, comparada con la masa del módulo (junto con la base), la temperatura cambia relativamente poco.